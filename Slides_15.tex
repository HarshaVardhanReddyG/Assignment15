\documentclass{beamer}
\usetheme{CambridgeUS}

\setbeamertemplate{caption}[numbered]{}

\usepackage{enumitem}
\usepackage{tfrupee}
\usepackage{amsmath}
\usepackage{amssymb}
\usepackage{textcomp, gensymb}
\usepackage{graphicx}
\usepackage{txfonts}

\def\inputGnumericTable{}

\usepackage[latin1]{inputenc}                                 
\usepackage{color}                                            
\usepackage{array}                                            
\usepackage{longtable}                                        
\usepackage{calc}                                             
\usepackage{multirow}                                         
\usepackage{hhline}                                           
\usepackage{ifthen}
\usepackage{caption} 
\providecommand{\mbf}{\mathbf}
\providecommand{\qfunc}[1]{\ensuremath{Q\left(#1\right)}}
\providecommand{\sbrak}[1]{\ensuremath{{}\left[#1\right]}}
\providecommand{\lsbrak}[1]{\ensuremath{{}\left[#1\right.}}
\providecommand{\rsbrak}[1]{\ensuremath{{}\left.#1\right]}}
\providecommand{\brak}[1]{\ensuremath{\left(#1\right)}}
\providecommand{\lbrak}[1]{\ensuremath{\left(#1\right.}}
\providecommand{\rbrak}[1]{\ensuremath{\left.#1\right)}}
\providecommand{\cbrak}[1]{\ensuremath{\left\{#1\right\}}}
\providecommand{\lcbrak}[1]{\ensuremath{\left\{#1\right.}}
\providecommand{\rcbrak}[1]{\ensuremath{\left.#1\right\}}}                                  
                               
\title{Assignment 15}
\author[CS21BTECH11017]{G HARSHA VARDHAN REDDY ( CS21BTECH11017 )}
\date{\today}
\logo{\large{AI1110}}


\begin{document}
% Title page frame
\begin{frame}
    \titlepage 
\end{frame}

% Remove logo from the next slides
\logo{}


% Outline frame
\begin{frame}{Outline}
    \tableofcontents
\end{frame}

%Question
\section{Problem Statement}
\begin{frame}{Problem Statement}
    \begin{block} {Papoulis Pillai Probability Random Variables and Stochastic Processes\\ 
    Exercise : 12-11}
    We wish to estimate the mean $\eta$ of a process $x(t) =\eta + v(t)$, where $R_{vv}(\tau) = 5\delta(\tau)$. Find the $0.95$ confidence interval of $\eta$ $?$ 
    \end{block}
\end{frame}
\section{Definitions}
\begin{frame}{Definitions}
    %\begin{block}{Auto-correlation function}
        %If $\{X_t\}$ is a \textbf{wide-sense stationary process} then the mean $\mu$  and the variance $\sigma ^{2}$ are time-independent.\\
    %and further the autocovariance function depends only on the lag between $t_{1}$ and $t_{2}$.\\
    %The autocovariance depends only on the time-distance between %the pair of values but not on their position in time.\\
    %The auto-correlation can be expressed as a function of the time-lag, and that this would be an even function of the lag $\{\tau =t_{2}-t_{1}\}$. This gives the more familiar forms for the auto-correlation function.
    %\begin{align}
    %   R_{XX}(t) &=E\left[X_{t+\tau}\Bar{X_t}\right]
    %\end{align}
    %\end{block}
    \begin{block}{Mean-Ergodic Processes}
The process is said to be mean ergodic if its time average $\eta_T$ tends
to the ensemble average $\eta$ as $T \to \infty$. 
    \end{block}
    \begin{block}{Auto covariance}
    For a stationary process,
    \begin{align}
        C(\tau) &= \frac{1}{2T}\int^{2T}_{-2T}C(\tau -\alpha)\left( 1-\frac{|\alpha|}{2T}\right)d\alpha \label{a}\\
    \end{align}
    Where,\\
    $C(\tau)$ is auto covariance of $x(t)$ 
    \end{block}
\end{frame}

\begin{frame}{Variance}
    We know that for a mean ergodic process
    \begin{align}
        \sigma_T^2&=C(0) \text{ and }\\
        C(\alpha) &=C(-\alpha)
    \end{align}
From \eqref{a},
\begin{align}
    \sigma_T^2=C(0)&= \frac{1}{2T}\int^{2T}_{-2T}C(\alpha)\left( 1-\frac{|\alpha|}{2T}\right)d\alpha\\
    \implies  \sigma_T^2&= \frac{1}{T}\int^{2T}_{0}C(\alpha)\left( 1-\frac{\alpha}{2T}\right)d\alpha
\end{align}
\end{frame}
\section{Solution}
\begin{frame}{Solution}
Given,\\
\begin{align}
    x(t) &= \eta +v(t)
\end{align}
Here,\\
$x(t)$ is mean ergodic process.\\
$v(t)$ is white noise with $R_{vv}(\tau)=5\delta(\tau)$\\
$C(\tau) = R_{vv}(\tau)$\\
As,
\begin{align}
    \sigma_{T}^2 &= \frac{1}{2T}\int_{-2T}^{2T} C(\alpha)\left(1-\frac{\alpha}{2T}\right)d\alpha
\end{align}
Therefore,\\
\begin{align}
    \sigma_T^2 &= \frac{1}{2T}\int_{-2T}^{2T} 5\delta(\tau)\left(1-\frac{\tau}{2T}\right)d\tau 
     =\frac{5}{2T}
\end{align}
\end{frame}
\begin{frame}{}
Therefore,
\begin{align}
    \sigma_T^2 &=\frac{5}{2T}
\end{align}
We have to find $\epsilon$ such that 
\begin{align}
    Pr(\eta - \varepsilon \leq \eta_T \leq \eta + \varepsilon) &= 0.95 \label{6}\\
    \implies Pr(|\eta_T-\varepsilon|\leq \eta) &= 0.95 \label{7}
\end{align}
As
\begin{align}
    Pr(|\eta_T-\varepsilon|\geq \eta) &= \int_{-\infty}^{-\eta-\varepsilon}f(x)dx+\int_{\eta + \varepsilon}^\infty f(x)dx \text{ and }\\
    \sigma_T^2 &= \int_{-\infty}^\infty (x-\eta)^2f(x)dx\\
    &\geq \int_{|\eta_T-\varepsilon|\geq \eta}(x-\eta)^2f(x)dx\\
    &\geq \varepsilon^2\int_{|\eta_T-\varepsilon|\geq \eta}f(x)dx
\end{align}
\end{frame}
\begin{frame}{}
    Therefore,
    \begin{align}
        \sigma_T^2 &\geq \varepsilon^2 \times Pr(|\eta_T-\varepsilon|\geq \eta)\\
        \implies Pr(|\eta_T-\varepsilon|\geq \eta) &\leq \frac{\sigma_T^2}{ \varepsilon^2} \label{13}
    \end{align} 

    From \eqref{7} and \eqref{13},
    \begin{align}
        Pr(|\eta_T-\varepsilon|\geq \eta) &= 1-0.95 \leq\frac{\sigma_T^2}{ \varepsilon^2} \\
        \implies\varepsilon^2 &\leq\frac{\sigma_T^2}{0.05}\label{eq}
    \end{align}
    Substituting $\sigma_T^2$ in \eqref{eq}
    \begin{align}
        \varepsilon^2 &\leq\frac{5}{2T\times 0.05}
    \end{align}
\end{frame}

\begin{frame}{}
    As $\varepsilon>0$
    \begin{align}
        \varepsilon^2 &\leq\frac{5}{2T\times 0.05}\\
        &\leq \frac{50}{T}\\
        \implies \varepsilon &\leq \sqrt{\frac{50}{T}}
    \end{align}
Therefore, $Pr(\eta - \varepsilon \leq \eta_T \leq \eta + \varepsilon) = 0.95 \implies \varepsilon \leq \sqrt{\frac{50}{T}}$
\end{frame}
\end{document}